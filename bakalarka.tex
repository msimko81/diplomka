\chapter{Matematické pojmy a značenia}

V nasledujúcom texte budeme používať toto značenie: 

Uvažovaný priestor bude euklidovský priestor $\mathcal E$ nad vektorovým 
priestorom $\mathcal V$.

Pojem \tu{tenzor} bude znamenať lineárne zobrazenie z
$\mathcal V$ do $\mathcal V$. Označme nasledovné množiny tenzorov: \[
\begin{split}
\lin &= \mbox{množina všetkých tenzorov,} \\
\linp &= \mbox{množina všetkých tenzorov $S$, pre ktoré je $\det S > 0$,} \\
\sym &= \mbox{množina všetkých symetrických tenzorov,} \\
\skw &= \mbox{množina všetkých antisymetrických tenzorov,} \\
\psym &= \mbox{množina všetkých symetrických, pozitívne definitných tenzorov,}
\\ \orth &= \mbox{množina všetkých ortogonálnych tenzorov,} \\
\orthp &= \mbox{množina všetkých rotácií.} \\
\end{split}
\]

Pojem \tu{telesa} $\mathcal B$ budeme používať na popis regulárnej  
oblasti v $\mathcal E$ a budeme naň odkazovať ako na \tu{referenčnú
konfiguráciu} $\mathcal B$. Body $\tu{p} \in \mathcal B$ nazývame \tu{materiálovými bodmi}.

\tu{Deformáciou} telesa $\mathcal B$ rozumieme hladké, bijektívne zobrazenie $f$, 
ktoré zobrazuje $\mathcal B$ na uzavretú oblasť v $\mathcal E$, a ktoré spĺňa
$\det \nabla f > 0$. Tenzor $F(\tu{p})=\nabla f(\tu{p})$ sa nazýva
\tu{deformačný gradient} a patrí do $\linp$.

Nech $\mathcal B$ je teleso. \tu{Pohybom} $\mathcal B$ nazveme zobrazenie 
$$x:\mathcal B \times \mathbb R \rightarrow \mathcal E$$ triedy $C^3$, 
pre ktoré je $x(\cdot,t)$ pre každý pevný čas $t$ deformáciou $\mathcal B$. 
\tu{Polohu} materiálového bodu $\tu{p}$ v čase $t$ označíme 
$$\tu{x}=x(\tu{p},t)$$ a $\mathcal B_t = x(\mathcal B,t)$ bude značiť priestor
vyplnený telesom v čase $t$. Definujeme \tu{trajektóriu} telesa ako množinu
$$\mathcal T=\{(\tu{x},t)|\,\tu{x}\in\mathcal B_t,t\in\mathbb R\}.$$

V každom čase $t$, $x(\cdot,t)$ má inverziu $$p(\cdot,t):\mathcal B_t \rightarrow
\mathcal B$$
tak, že $$x(p(\tu{x},t),t)=\tu{x}, \qquad p(x(\tu{p},t),t)=\tu{p}.$$

Pre dané $(\tu{x},t) \in \mathcal T$, je $$\tu{p}=p(\tu{x},t)$$ materiálový
bod, ktorý má v čase $t$ polohu $\tu{x}$. Zobrazenie $$p:\mathcal
T\rightarrow\mathcal B$$ nazývame \tu{referenčným zobrazením} pohybu.

\tu{Materiálovým poľom} rozumieme zobrazenie definované na $\mathcal
B\times\mathbb R$; a \tu{priestorovým poľom} zobrazenie definované na $\mathcal
T$. Môžme transformovať materiálové pole na priestorové a naopak. Definujeme
\tu{priestorový popis} $\Phi_s$ materiálového poľa 
$(\tu{p},t)\rightarrow\Phi(\tu{p},t)$ ako $$\Phi_s(\tu{x},t)=\Phi(p(\tu{x},t),t),$$
a \tu{materiálový popis} $\Omega_m$ priestorového poľa
$(\tu{x},t)\rightarrow\Omega(\tu{x},t)$ ako $$\Omega_m(\tu{p},t)=\Omega(x(\tu{p},t),t).$$

Pre dané materiálové pole $\Phi$ značíme 
$$\dot\Phi(\tu{p},t)=\ddt \Phi(\tu{p},t)$$
deriváciu vzhľadom k času $t$ pre pevný bod $\tu{p}$, a
$$\nabla\Phi(\tu{p},t)=\nabla_p\,\Phi(\tu{p},t)$$
gradient vzhľadom k $\tu{p}$ pre $t$ pevné.

Podobne, pre priestorové pole $\Omega$ značíme 
$$\Omega'(\tu{x},t)=\ddt \Omega(\tu{x},t)$$
deriváciu vzhľadom k času $t$ pre pevnú polohu $\tu{x}$, a
$$\grad\Omega(\tu{x},t)=\nabla_x\,\Omega(\tu{x},t)$$
značí gradient vzhľadom k $\tu{x}$ pre pevné $t$.

Definujeme \tu{materiálovú časovú deriváciu} $\dot\Omega$ priestorového poľa
$\Omega$ ako $$\dot\Omega=((\Omega_m)^.)_s\,;$$ t.j.,
$$\dot\Omega(\tu{x},t)=\ddt \Omega(x(\tu{p},t),t)|_{\tu{p}=p(\tu{x},t)}\,.$$

Ďalej, definujeme \tu{priestorovú divergenciu} $\dv$ ako operáciu divergencie
pre priestorové pole tak, že $\grad$ je príslušný gradient. Teda,
pre priestorové vektorové pole $v$, máme 
$$\dv v(\tu{x},t)=\tr\grad v(\tu{x},t).$$

\tu{Rýchlosťou} materiálového bodu $\tu{p}$ rozumieme $$\dot x(\tu{p},t)=\ddt
x(\tu{p},t)$$ a $v:\mathcal T\rightarrow\mathcal V$ definované predpisom
$$v(\tu{x},t)=\dot x(p(\tu{x},t),t)$$ 
je \tu{priestorový popis rýchlosti}. 
Priestorové pole $$L=\grad v$$ sa nazýva \tu{gradientom rýchlosti}. Označíme
$$L=D+W,$$ kde $D$ a $W$, značia symetrickú a antisymetrickú časť $L$.

Použitím predchádzajúcej definície gradientu rýchlosti sa dá ukázať, že 
$\dot F=L_m F$ je materiálová časová derivácia deformačného gradientu $F$.

\tu{Systémom síl} pre $\mathcal B$ počas pohybu (s trajektóriou $\mathcal T$), 
rozumieme dvojicu $(s,b)$ zobrazení
$$s:\mathcal N\times\mathcal T\rightarrow\mathcal V, \qquad b:\mathcal T\rightarrow\mathcal V,$$
kde $\mathcal N$ je množina všetkých jednotkových vektorov 
z $\mathcal V$\footnote{Presnejšia definícia je v Gurtin \cite[s. 99]{gurtin}.}.

Podľa Cauchyho vety\footnote{Viď Gurtin \cite[s. 101]{gurtin}.}, 
existuje priestorové tenzorové pole $T$ (nazvané \tu{Cauchyho napätie}) tak, že 
\begin{itemize}
 \item $s(n)=Tn$ pre každý jednotkový vektor $n$,
 \item $T$ je symetrické,
 \item $T$ spĺňa \tu{rovnicu pohybu} $$\dv T+b=\rho\dot v,$$  
\end{itemize}
kde $\rho$ je hustota pohybu.

\tu{Dynamickým procesom} rozumieme dvojicu $(x,T)$, kde 
\begin{itemize}
 \item $x$ je pohyb,
 \item $T$ je symetrické tenzorové pole na trajektórii $\mathcal T$ pohybu x,
 \item $T(\tu{x},t)$ je hladké zobrazenie $\tu{x}$ do $\mathcal B_t$.
\end{itemize}

\tu{Materiálové teleso} je teleso $\mathcal B$ spolu s množinou $\mathcal C$ 
dynamických procesov. $\mathcal C$ nazývame \tu{konštitučnou triedou} telesa.

Nech $x$ a $x^*$ sú pohybmi telesa $\mathcal B$. 
Vravíme, že $x$ a $x^*$ sú {\it spojené zmenou pozorovateľa}, ak
\eq{gu20-1}{x^*(\tu{p},t)=q(t)+Q(t)[x(\tu{p},t)-o]}
pre každý materiálový bod $\tu{p}$ a čas $t$, kde $q(t)$ je bod v priestore a
$Q(t)$ je rotácia. 

Pre $$L=\grad v, \qquad L^*=\grad v^*,$$ kde 
$$v=(\dot x)_s, \qquad v^*=(\dot x^*)_s,$$ dostaneme
$$L^*=QLQ^T+\dot{Q}Q^T,\qquad D^*=QDQ^T,$$ kde $D$ a $D^*$ sú symetrické
časti $L$ a $L^*$. Máme tak $\tr L^*=\tr L$.

Vravíme, že dvojica dynamických procesov $(x,T)$ a $(x^*,T^*)$ 
\tu{je spojená zmenou pozorovateľa}, ak existujú $C^3$ zobrazenia 
$$q:\mathbb R\rightarrow\mathcal E, \qquad Q:\mathbb R\rightarrow\orthp$$
tak, že
\begin{itemize}
	\item \eref{gu20-1} platí pre všetky $\tu{p}\in\mathcal B$ a $t\in\mathbb R$,
	\item $T^*=QTQ^T$ je trajektóriou $x$.
\end{itemize}

Vravíme, že \tu{odozva materiálového telesa je nezávislá na pozorovateľovi}, ak
má príslušná konštitučná trieda $\mathcal C$ nasledujúcu vlastnosť: ak proces
$(x,T)$ patrí do $\mathcal C$, potom tam patrí aj každý dynamický proces spojený
s $(x,T)$ zmenou pozorovateľa.

\chapter{Rovnice pohybu}

V tejto práci riešime prúdenie stlačiteľnej Newtonovskej tekutiny, čo je
materiál, pre ktorý je tenzor Cauchyho napätia definovaný konštitučnou rovnicou tvaru
\eq{gu22-1}{T = -\pi I + C[L],}
kde $C$ je lineárna funkcia gradientu rýchlosti $$L=\grad v.$$

V knihe Gurtin \cite[s. 147]{gurtin}, sa pod pojmom Newtonovská tekutina rozumie {\it
nestlačiteľná} Newtonovská tekutina. Navier-Stokesové rovnice sú odvodené s
predpokladom $\tr L=0$, ktorý znamená nestlačiteľnosť. V našom prípade
potrebujeme zahrnúť efekty stlačiteľnosti a člen $\tr L=\dv v$ nemôžme
zanedbať. Pod pojmom Newtonovská tekutina budeme rozumieť {\it stlačiteľnú} Newtonovskú
tekutinu.
Z dôvodu zjednodušenia konštitučnej rovnice definujme \tu{extra napätie} $T_0$ vzťahom
$$T_0=T+\pi I=T-\fr{1}{3}(\tr T)I.$$
Konštitučná rovnica \eref{gu22-1} tak nadobudne jednoduchý tvar
\eq{gu22-3}{T_0=C[L].}

V súlade s predchádzajúcim, \tu{Newtonovskou tekutinou} rozumieme
stlačiteľné materiálové teleso spolu s nasledujúcou konštitučnou rovnicou: 
existuje lineárna {\it funkcia odozvy} $$C:\lin \rightarrow \sym$$ tak, že 
konštitučná trieda $\mathcal C$ je množinou všetkých dynamických procesov $(x,T)$ 
spĺňajúcich konštitučnú rovnicu \eref{gu22-3}.

V nasledújúcej vete ukážeme, že funkcia odozvy je určená {\it dvoma konštantami}. 

\tu{Veta}\footnote{Viď Gurtin \cite[s. 149]{gurtin}.}
Nutná a postačujúca podmienka aby odozva Newtonovskej tekutiny bola
nezávislá na zmene pozorovateľa je, že funkcia odozvy $C$ má tvar
\eq{gu22-4}{C[L]=2\mu D + \lambda (\tr L)I}
pre každé $L \in \lin$, kde $$D=\fr{1}{2}(L+L^T).$$
Skalárne konštanty $\mu$ a $\lambda$ sú takzvané \tu{koeficienty viskozity} tekutiny.
\begin{proof}
Použijeme dôkaz uvedený v Gurtinovi s tým, že v ňom zahrnieme efekty stlačiteľnosti. 

Predpokladajme, že platí \eref{gu22-4}. Nech $(x,T)$ patrí do konštitučnej
triedy $\mathcal C$ tekutiny. Potom $$T_0=2\mu D+\lambda (\tr L)I.$$
Nech ďalej $(x^*,T^*)$ je spojené s $(x,T)$ zmenou pozorovateľa. Potom
$$T^*=QTQ^T,\qquad D^*=QDQ^T,$$
a
$$\tr T^*=\tr(QTQ^T)=\tr T.$$
Pre extra napätie $T_0^*$ máme 
\[
\begin{split}
T^*_0&=T^*-\fr{1}{3}(\tr T^*)I=QTQ^T-\fr{1}{3}(\tr T)QQ^T=QT_0Q^T\\
&=Q(2\mu D)Q^T + \lambda\,\tr(QLQ^T)I = 2\mu D^* + \lambda(\tr L^*)I.
\end{split}
\]
V predchádzajúcom výraze sme využili rovnosť
$$L^*=QLQ^T+\dot{Q}Q^T$$ a pretože $\dot{Q}Q^T \in \skw$, máme
$$\tr L^*=\tr(QLQ^T)=\tr L,$$

Teda $(x^*,T^*)\in \mathcal C$ a odozva je nezávislá na pozorovateľovi. Tým máme
ukazané, že podmienka \eref{gu22-4} je postačujúca.

Dôkaze nutnosti je založený na nasledujúcej lemme a vete o reprezentácii. 

\tu{Lemma.} Nech $L \in \lin$ je konštantný tenzor. 
Potom existuje pohyb $x$ s gradientom rýchlosti rovným \eq{gu22-6}{\grad v=L.}
\begin{proof}
Uvažujme $$F(t)=e^{Lt}$$
kde $F$ je jediné riešenie
\eq{gu22-7}{\dot F=LF, \qquad F(0)=I.}
Teda
$$x(\tu{p},t)=\tu{q}+F(t)[\tu{p}-\tu{q}]$$
definuje pohyb s deformačným gradientom $F$. Ďalej, \eref{gu22-6}
vyplýva z \eref{gu22-7}$_1$, pretože $\dot F=(\grad v)_mF$ a $L=L_m$.
\end{proof}

\tu{Veta o reprezentácii pre izotropické tenzorové funkcie.}\footnote{Viď Gurtin
\cite[s. 235]{gurtin}.} 
Lineárna funkcia
$$  G:Sym \rightarrow Sym $$
je izotropická práve vtedy, keď existujú skalárne konštanty $\mu$ a $\lambda$
také, že
\eq{gu37-22}{G(A)=2\mu A + \lambda (\tr A)I}
pre každé $A \in \sym$.

Vráťme sa k dôkazu predchádzajúcej vety. K ukázaniu 
{\it nutnosti} \eref{gu22-4} predpokladajme, že
\eq{gu22-8}{\mbox{odozva je nezávislá na pozorovateľovi.}}
Nech $L \in \lin$ je ľubovoľné, nech ďalej $x$ je pohyb zostrojený v
predchádzajúcej lemme, a nech $T=T_0=C[L]$ je konštantné pole definované
rovnosťou \eref{gu22-3}. Je teda $(x,T)\in \mathcal C$. 
Nech $(x^*,T^*)$ je spojené s $(x,T)$ zmenou pozorovateľa. 
Potom podľa \eref{gu22-8} je $(x^*,T^*)\in \mathcal C$ a
\eq{gu22-9}{T^*_0=C[L^*].}
Keďže
$$T^*_0=QT_0Q^T,\qquad L^*=QLQ^T+\dot QQ^T;$$
zo vzťahu \eref{gu22-9} máme
$$QT_0Q^T=C[QLQ^T+\dot QQ^T],$$
a použitím \eref{gu22-3} a \eref{gu22-6} dostaneme
\eq{gu22-10}{QC[L]Q^T=C[QLQ^T+\dot QQ^T].}
Táto rovnosť platí zrejme pre každé $L\in\lin$ (definičný obor $C$) a pre každú
hladkú funkciu $Q:\mathbb R \rightarrow \orthp$ triedy $C^3$. Uvažujme $L$ pevné a
definujme $$Q(t)=e^{-Wt},$$ kde $$W=\fr{1}{2}(L-L^T).$$
Potom, pretože $W$ je antisymetrické, je $Q(t)$ rotácia, a $$Q(0)=I,\qquad \dot
Q(0)=-W.$$ Využitím takto definovanej funkcie $Q$ v rovnici \eref{gu22-10} pre 
$t=0$ dostaneme $$C[L]=C[L-W]=C[D],$$ kde $$D=\fr{1}{2}(L+L^T).$$
Teda $C$ je kompletne definované svojou reštrikciou na $\sym$.
Ďalej, nech $Q$ je konštantná funkcia s hodnotami v $\orthp$. Potom
z \eref{gu22-10} pre $L=D$ ($D\in\sym$) dostávame $$QC[D]Q^T=C[QDQ^T].$$
Pretože táto rovnosť platí pre každé $D\in\sym$ a každé $Q\in\orthp$, 
reštrikcia $C$ na $\sym$ je izotropická. Z reprezentácie \eref{gu37-22} tak dostávame
$$C[D]=2\mu D + \lambda (\tr L)I$$ pre každé $D\in\sym$. \\
Podľa \eref{gu22-4}, konštitučná rovnica \eref{gu22-1} dostane tvar
\eq{gu22-13}{T=-\pi I+2\mu D+\lambda (\tr L)I.}
\end{proof}


Uvažujme rovnicu pohybu v tvare
$$\rho[v'+(\grad v)v]=\dv T+b.$$
a dosaďme za $T$ podľa \eref{gu22-13}.
Dostávame,
$$2\,\dv D=\dv(\grad v+\grad v^T)=\Delta v+\grad \dv v$$
a
$$\dv(\tr L)I=\grad(\tr L)=\grad \dv v,$$
kde $\Delta=\dv \grad$ je priestorový laplacian. Rovnica pohybu tak prejde v tvar
\eq{gu22-14}
{
\rho[v'+(\grad v)v]=\mu\Delta v+(\lambda+\mu)\,\grad \dv v -\grad \pi + b.
}

Tieto (vektorové) rovnice sa nazývajú \tu{Navier-Stokesove rovnice}; pre dané $\mu$,
$\lambda$ a $b$ 
predstavujú nelineárny systém parciálnych diferenciálnych rovníc pre rýchlosť
$v$, hustotu $\rho$ a tlak $\pi$.
Dané rovnice doplníme o rovnicu kontinuity v tvare
\eq{si-eq-cont}
{
\rho'+\dv(\rho v)=0.
}

Máme tak štyri rovnice pre päť neznámych, preto musíme dodať ešte jednu ďalšiu
rovnicu. Budeme uvažovať barotropické prúdenie\footnote{Viď Feistauer et al. \cite[s.
33]{feistauer}.}, pre ktoré je tlak definovaný známou funkciou hustoty
\eq{si-bar-flow}
{
\pi=\widehat\pi(\rho).
}

V ďalšom budeme predpokladať, že $\mu>0,\, \lambda+\mu>0$. Druhá nerovnosť je
splnená napr. v prípade $\lambda=-\frac{2}{3}\mu$, ktorý sa obvykle používa v
technickej praxi.

\chapter{Formulácia úlohy}

V nasledujúcom texte sa budeme zaoberať dvoj-dimenzionálnym modelom popisujúcim
interakciu viskóznej, stlačiteľnej tekutiny a leteckého profilu. Profil bude predstavovať
tuhé teleso s dvoma stupňami voľnosti - vertikálnymi a torznými vibráciami (viď \iref{si-im-airfoil}). 
Rovnice popisúce pohyb profilu uvedieme neskôr.

\begin{figure}[h]
  \begin{center}
    \img{profile.png}
    \caption{Model leteckého profilu}
    \label{si-im-airfoil}
  \end{center}
\end{figure}

\begin{figure}[h]
  \begin{center}
    \img{problem-setting.png}
    \caption{Zadanie úlohy}
    \label{si-im-setting}
  \end{center}
\end{figure}

Daná úloha má časovo závislú hranicu (pohybujúci sa profil) a teda aj časovo
závislú výpočtovú oblasť (viď \iref{si-im-setting}).

\section{Vstupné dáta úlohy}

Uvažujme úlohu danú na oblasti
$$\widetilde\Omega:=\bigcup_{t\in[0,T]}\Omega_t\times\{t\}.$$
Rozdeľme hranicu oblasti na štyri časti. Tri z nich budú časovo nezávislé,
štvrtá predstavuje pohybujúci sa letecký profil a na čase závisí:
\[
\begin{array}{ll}
\Gamma_I:=\Gamma_I\times[0,T] &\qquad \mbox{vstup,} \\
\Gamma_O:=\Gamma_O\times[0,T] &\qquad \mbox{výstup,} \\
\Gamma_W:=\Gamma_W\times[0,T] &\qquad \mbox{myslená stena prúdenia,} \\
\Sigma:=\bigcup_{t\in[0,T]}\Gamma_{W_t}\times\{t\} &\qquad \mbox{letecký profil.}
\end{array}
\]
V oblasti $\widetilde\Omega$ uvažujeme Navier-Stokesove rovnice, rovnicu
kontinuity a rovnicu pre tlak barotropickej tekutiny:
\eq{si-problem-domain}
{
\setlength\arraycolsep{2pt}
\begin{array}{cl}
\rho[v'+(\grad v)v] = \mu\Delta v+(\lambda+\mu)\,\grad \dv v -\grad \pi + b
  & \qquad\mbox{v } \widetilde\Omega, \\
\rho'+\dv(\rho v) = 0 & \qquad\mbox{v } \widetilde\Omega, \\
\pi = \widehat\pi(\rho) & \qquad\mbox{v } \widetilde\Omega.
\end{array}
}
Okrajové podmienky pre časovo nezávislú časť hranice:
\eq{si-problem-boundary}
{
\setlength\arraycolsep{2pt}
\begin{array}{rll}
v &= v_D & \qquad\mbox{na } \Gamma_I \cup \Gamma_W, \\
-(\pi-\pi_{ref})n+\mu\,(\grad v)n+(\lambda+\mu)(\dv v)n &= 0 & \qquad\mbox{na } \Gamma_O, \\
\rho &= \rho_D & \qquad\mbox{na } \Gamma_I.
\end{array}
}
Počiatočné podmienky:
\eq{si-problem-initial}
{
\setlength\arraycolsep{2pt}
\begin{array}{rll}
v(x,0) &= v_0(x) & \qquad\mbox{v } \Omega_0, \\
\rho(x,0) &= \rho_0(x) & \qquad\mbox{v } \Omega_0.
\end{array}
}
Potrebujeme tiež predpísať okrajové podmienky na časti hranice $\Sigma$ a
počiatočné podmienky na $\Omega_0$. Tomu sa budeme venovať neskôr.

Skutočnosť, že oblasť vyplnená tekutinou sa s časom mení, spôsobuje určité
obtiaže. Môžme ich vyriešiť použitím Arbitrary Lagrangian-Eulerian (ALE)
formulácie pre matematický popis úlohy s pohyblivou hranicou. 

\section{Rovnice pre pohyb profilu}

V našej úlohe môže profil vykonávať dva druhy pohybu: vertikálne a torzné
vibrácie. Tieto sú popísané dvoma stupňami voľnosti: uhlom náklonu profilu
$\alpha$ a vertikálnou výchylkou $h$. Vývoj hodnôt týchto veličín pre malé
výchylky profilu je popísaný diferenciálnymi rovnicami
\footnote{Viď Růžička \cite[s. 17]{ruzicka}.} 
\eq{si-airfoil-description}
{
\setlength\arraycolsep{2pt}
\begin{array}{c}
m \ddot h+D_{hh}\dot h+D_{h\alpha}\dot\alpha+S_\alpha\ddot\alpha+k_{hh}h=-L_2, \\
I_\alpha\ddot\alpha+D_{\alpha h}\dot h+D_{\alpha\alpha}\dot\alpha+S_\alpha\ddot h+k_{\alpha\alpha}\alpha=M.
\end{array}
}
V uvedených rovniciach používame nasledovné označenie:
\[
\setlength\arraycolsep{2pt}
\begin{array}{ll}
m=\int_{\Pi_t}{\tilde\rho}\dV & \mbox{hmotnosť profilu,} \\
S_\alpha=\int_{\Pi_t}{x\tilde\rho}\dV & \mbox{statický moment,} \\
I_\alpha=\int_{\Pi_t}{x^2\tilde\rho}\dV & \mbox{moment zotrvačnosti,} \\
L_2=-\int_{\Gamma_{W_t}}\sum_{j=1}^2T_{2j}n_j\dA & \mbox{aerodynamický vztlak,} \\
M=-\int_{\Gamma_{W_t}}\sum_{i,j=1}^2T_{ij}n_j r_i^{ort})\dA \quad
&\mbox{aerodynamický moment,}
\end{array}
\]
kde $\Pi_t$ je plocha profilu, $\tilde\rho$ je hustota profilu, 
$\Gamma_{W_t}=\partial\Pi_t$, $T$ je tenzor napätia získaný z \eref{gu22-13},
$r_1^{ort}=-(x_2-x_{EA2})$ a $r_2^{ort}=x_1-x_{EA1}$. Ďalej,
\[
\begin{array}{ll}
k_{hh} & \mbox{vertikálna tuhosť,} \\
k_{\alpha\alpha} & \mbox{torzná tuhosť,} \\
D_{hh},D_{h\alpha},D_{\alpha h},D_{\alpha\alpha} \quad & \mbox{koeficienty
viskózneho tlmenia}
\end{array}
\]
sú dané (konštantné) parametre.

Rovnice \eref{si-airfoil-description} doplníme týmito počiatočnými podmienkami \[
\begin{array}{ll}
\alpha(0)=\alpha_0, & \dot\alpha(0)=\alpha_1, \\
h(0)=h_0, & \dot h(0)=h_1. \\
\end{array}
\]

\begin{figure}[t]
  \begin{center}
    \img{vibrations.png}
    \caption{Vibrácie profilu}
    \label{si-im-vibrations}
  \end{center}
\end{figure}


\section{ALE formulácia}

Uvažujeme Navier-Stokesove rovnice v pohybujúcej sa oblasti 
$\widetilde\Omega=\Omega_t\times[0,T]$ (presnejšie 
$\bigcup_{t\in[0,T]}\Omega_t\times\{t\}$). \\
Za účelom simulácie prúdenia tekutiny v pohybujúcej sa oblasti použijeme 
{\it Arbitrary Lagrangian-Eulerian} (ALE) metódu\footnote{Viď napr. Quarteroni 
\cite[s. 37]{quarteroni}; Sváček \cite[s. 6]{svacek}.}. 

Nech $\Omega_0$ je počiatočná (referenčná) oblasť a $\Omega_t$ výpočtová oblasť v
(neskoršom) čase $t$. Zavedieme ALE zobrazenie \[
\begin{split}
&\mathcal A_t:\Omega_0\rightarrow\Omega_t\\
&X\mapsto y=y(X,t)=\mathcal A_t(X),
\end{split}
\]
ktoré zobrazuje referenčnú oblasť $\Omega_0$ na výpočtovú oblasť $\Omega_t$ tak,
že $\mathcal A_t$ je spojite diferencovateľné a bijektívne na $\Omega_0$. 

Definujeme pole {\it rýchlosti pohybu oblasti} v bodoch $X$ referenčnej oblasti pre
každú časovú vrstvu $t$ $$\tilde w(X,t)=\ddt y(X,t)=\ddt \mathcal A_t(X),$$
ktorá má v priestorových premenných tvar
$$w=\tilde w\circ A_t^{-1}, \qquad \mbox{i.e.} \quad w(y,t)=\tilde w(\mathcal A_t^{-1}(y),t).$$

Pre funkciu $f:\widetilde\Omega\rightarrow\mathbb R$ definujeme {\it ALE
deriváciu} $f$ vzťahom $$\DADt f(y,t) = \ddt \tilde f(X,t),$$
kde $\tilde f=f\circ\mathcal A_t$ a $X=\mathcal A_t^{-1}(y)$.

Použitím reťazového pravidla pre deriváciu dostaneme
\[
\begin{split}
\DADt f(y,t) &= \ddt f(\mathcal A_t(X),t) \\
&= \ddt f(y,t) + \grad f(y,t) \cdot \ddt \mathcal A_t(X)|_{X=\mathcal A_t^{-1}(y)} \\
&= \ddt f(y,t) + \grad f(y,t) \cdot w(y,t).
\end{split}
\]

Využitím ALE derivácie môžme prepísať Navier-Stokesove rovnice do tvaru
\eq{si-ALE-form}
{
\setlength\arraycolsep{2pt}
\begin{array}{c}
\rho[\DADt v+(\grad v)(v-w)] = \mu\Delta v+(\lambda+\mu)\,\grad \dv v -\grad \pi + b \\
\DADt \rho+\dv(\rho v)-\grad\rho \cdot w = 0 \\
\pi = \widehat\pi(\rho)
\end{array}
}
kde všetky rovnice uvažujeme v oblasti $\widetilde\Omega$. 
Poznamenajme, že rovnica kontinuity môže byť zapísaná v tvare
\eq{si-ALE-cont}{\DADt \rho+\rho\,\dv(v)+\grad\rho \cdot (v-w) = 0.}
Okrajové a počiatočné podmienky ostanú nezmenené. 
Poznamenajme, že predpokladáme $\mu>0,\,\lambda+\mu>0$.


\section{Slabá formulácia}

Definujme najprv priestory testovacích funkcií. Nech $q\in Q\!=\!L^2(\Omega_t)$ \\
a $u\in V\!=\!\{u \in H^1(\Omega_t)^2 : u|_{\Gamma_D}=0\}$, kde 
$\Gamma_D=\Gamma_I \cup \Gamma_W \cup \Gamma_{W_t}$ je časť hranice, na ktorej
predpisujeme Dirichletovu okrajovú podmienku.

Vynásobením rovnice \eref{si-ALE-form}$_1$ ľubovoľnou $u\in V$,
integrovaním cez $\Omega_t$ a použitím Greenovej vety dostaneme
\[
\begin{split}
\int_{\Omega_t}&{\rho\; \DADt v \cdot u}\dV
+ \int_{\Omega_t}{\rho(\grad v)(v-w) \cdot u}\dV = \\
&-\mu \int_{\Omega_t}{\grad v \cdot \grad u}\dV
- (\lambda+\mu) \int_{\Omega_t}{\dv v \,\dv u}\dV \\
&+ \int_{\Omega_t}{\pi \,\dv u}\dV 
+ \int_{\Omega_t}{b \cdot u}\dV \\
&+ \int_{\Gamma_O}{[-\pi + \mu\,(\grad v) + (\lambda+\mu)(\dv v)]\,n\cdot u}\dA
\end{split}
\]

Podobným postupom s rovnicou \eref{si-ALE-cont}, kde za testovaciu funkciu volíme
$q\in Q$, dostaneme
\[
\int_{\Omega_t}{\DADt \rho \;q}\dV
+\int_{\Omega_t}{\rho \,\dv v \;q}\dV
+\int_{\Omega_t}{\grad \rho \cdot (v - w) \;q}\dV
=0
\]

Pre zjednodušenie zápisuje definujme lineárne formy\footnote{Označenie z
Feistauer et al. \cite[s. 368]{feistauer}.}: 
\[
\begin{split}
a(v,u) &= \mu\,(\grad v,\grad u) + (\lambda+\mu)(\dv v,\dv u), \\
b(u,q) &= (\dv u,q), \\
\alpha(v,\rho,q) &= (v \cdot \grad \rho,q), \\
d(\rho,w,v,u) &= (\rho(\grad v)w,u), \\
e(\rho,v,q) &= (\rho\,\dv v,q). \\
\end{split}
\]
Použitím takto definovaných foriem prepíšeme pôvodné rovnice do tvaru
\eq{si-weak-homog}{
\begin{split}
(\rho\; \DADt &v,u) + d(\rho,v-w,v,u) + a(v,u) \\
&=b(u,\pi) + (b,u) + \int_{\Gamma_O}{\pi_{ref}\,n \cdot u}\dA, \\
(\DADt \rho,q) &+ e(\rho,v,q) + \alpha(v-w,\rho,q) = 0,
\end{split}
}
kde sme využili takzvanú slabú okrajovú podmienku \eref{si-problem-boundary}$_2$.

\section{Okrajové podmienky}

Predpokladajme, že pre každé $t\in[0,T]$ existuje $v^* \in H^1(\Omega_t)^2$
tak, že
\[
\begin{array}{ll}
v^*(x,t)=v_D(x,t), &x \in \Gamma_I \cup \Gamma_W \\
v^*(x,t)=w(x,t), &x \in \Gamma_{W_t}
\end{array}
\]
(v zmysle stôp). Potom je {\it ``slabá formulácia''} úlohy nasledová:\\
\begin{itemize}
 \item Nájsť $v$ tak, že $v - v^* \in V$; $\rho \in Q$
 \item rovnica \eref{si-weak-homog}$_1$ je splnená $\forall u \in V$. 
\end{itemize}

Okrajová podmienka pre hustotu $\rho$ predpísana na vstupe $\Gamma_I$ je
formulovaná v takzvanom slabom integrálnom zmysle\footnote{Viď Feistauer et al.
\cite[s. 373]{feistauer}.} 
\eq{si-weak-boundary-cond}{
\begin{split}
(\DADt &\rho,q) + e(\rho,v,q) + \alpha(v-w,\rho,q) 
- \gamma \int_{\Gamma_I}{\rho v_D \cdot nq}\dA = \\
&- \gamma \int_{\Gamma_I}{\rho_D v_D \cdot nq}\dA \qquad \forall q\in Q,
\end{split}
}
kde $\gamma$ je vhodný parameter.

Vo vyššie uvedenej formulácii predpokladáme, že funkcie $\rho,\,v,\,b,\,w,\rho_D,\,v_D$ sú
natoľko regulárne, aby formy vystupujúce v \eref{si-weak-homog} a
\eref{si-weak-boundary-cond} mali zmysel.

\section{Diskrétna úloha}

Predpokladajme pre jednoduchosť, že oblasti $\Omega_t$ sú polygonálne.
Nech $\{\mathcal T_h\}_{h \in (0,T)}$ je regulárny systém triangulácií oblasti 
$\widetilde\Omega:=\bigcup_{t\in[0,T]}\Omega_t\times\{t\}$. Na časovom intervale
$[0,T]$ zvoľme delenie $t_n=n\tau, n=0,\ldots,r$ pre časový krok~$\tau$. 
Pre ALE deriváciu funkcie $f$ definovanej na $\widetilde\Omega$ máme 
\[
\begin{split}
\DADt f(y_n,t_n) &= \ddt\tilde{f}(X,t_n) \\
&\approx(\tilde{f}(X,t_n)-\tilde{f}(X,t_{n-1}))/\tau \\
&= (f(y_n,t_n)-f(y_{n-1},t_{n-1}))/\tau, 
\end{split}
\]
kde $y_n=\mathcal A_{t_n}(X),\,y_{n-1}=\mathcal A_{t_{n-1}}(X)$.

Z dôvodu zjednodušenia zápisu budeme písať $f^n=f(y_n,t_n)$ a $\dadt
f^n=(f^n-f^{n-1})/\tau$.

Približné riešenie budeme hľadať na každej časovej vrstve $t_n$ v
konečnorozmernom priestore konečných elementov $X_h$ a $Q_h$. \\
Položíme $Q_h=X_h^{(m)}$, $X_h=[X_h^{(k)}]^2$, $V_h=\{v_h \in [X_h^{(k)}]^2;\,
v_h|_{\Gamma_D}=0\}$, kde $X_h^{(p)}=\{v_h\in C(\bar{\Omega}_h);\, v_h|_K \in
P^p(K) \;\forall K\in \mathcal T_h \}$ a $P^p(K)$ je množina všetkých polynómov
na $K$ stupňa $\leq p$. 


Aproximujme priestory $V$ a $Q$ pomocou $V_h$ a $Q_h$. Použijeme aproximácie 
\[
\begin{split}
v^n &\approx v_h^n \in V_h, \\
\rho^n &\approx \rho_h^n \in Q_h, \\
\DADt v^n &\approx (v^n-v^{n-1})/\tau \approx (v_h^n-v_h^{n-1})/\tau = \dadt
v_h^n, \\
\DADt \rho^n &\approx (\rho^n-\rho^{n-1})/\tau \approx
(\rho_h^n-\rho_h^{n-1})/\tau = \dadt \rho_h^n 
\end{split}
\]

Definujme ďalej funkciu
$$q_h+\delta q_{h\beta} \quad \mbox{s } q_{h\beta}=(v_h^{n-1}-w_h^{n-1})\cdot\grad{q_h}$$ 
pre vhodnú konštantu $\delta>0$, ktorá bude použitá namiesto $q_h$ z dôvodu
zamedzenia Gibbsovho javu v numerickom riešení (takzvaná streamline diffusion
test function)\footnote{Viď Feistauer et al. \cite[s. 346]{feistauer}}. 

Nech $v_h^* \in X_h$ je aproximácia $v^*$. Môžme použiť aproximáciu 
\[
\begin{array}{ll}
v_h^*(P_i,t)=v_D(P_i,t) \quad &\forall P_i\in\Gamma_I\cup\Gamma_W, \\
v_h^*(P_i,t)=w(P_i,t) \quad &\forall P_i\in\Gamma_{W_t}, \\
v_h^*(P_i,t)=0 \quad &\forall P_i\in\Omega_t. \\
\end{array}
\]


Dostávame tak formuláciu diskrétnej úlohy:\\
Nájsť $v_h^n \in X_h$ tak, že $v_h^n - v_h^{*n} \in V_h$; $\rho_h^n \in Q_h$
spĺňajúce nasledujúce rovnice:

\eq{si-discrete-problem}{
\begin{split}
(\rho_h^{n-1}\; &\dadt v_h^n,u_h) + d(\rho_h^{n-1},v_h^{n-1}-w_h^{n-1},v_h^n,u_h) + a(v_h^n,u_h) \\
&=b(u_h,\pi_h^{n-1}) + (b_h^{n-1},u_h) + \int_{\Gamma_O}{\pi_{ref}\,u_h \cdot
n}\dA \qquad \forall u_h \in V_h, \\ 
(\dadt &\rho_h^n,q_h) + e(\rho_h^{n-1},v_h^n,\sdtf) \\
&+\alpha(v_h^{n-1}-w_h^{n-1},\rho_h^n,\sdtf) 
- \gamma \int_{\Gamma_I}{\rho_h^n v_D^n \cdot nq_h}\dA \\
&= -\gamma \int_{\Gamma_I}{\rho_D^n v_D^n \cdot nq_h}\dA \qquad\forall q_h \in Q_h, \\
&\pi_h^n = \widehat\pi(\rho_h^n).
\end{split}}

Môžme písať 
$$v_h^n=v_h^{*n}+z_h^n, \quad \mbox{kde } z_h^n \in V_h.$$
Pri predpoklade, že $u_h^{n-1},\rho_h^{n-1},\pi_h^{n-1},w_h^{n-1}$ sú známe,
použitím substitúcie pre $v_h^n$ dostávame lineárny systém pre parametre
určujúce neznáme funkcie $z_h^n$ a $\rho_h^n$.

Systém \eref{si-discrete-problem} môžme riešiť v dvoch oddelených krokoch.
Najprv nájdeme $v_h^n$ riešením prvej rovnice. Využitím výsledku nájdeme
$\rho_h^n$ riešením druhej.
