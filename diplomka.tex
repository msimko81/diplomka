\chapter{Existencia riešenia diskrétneho problému}
 
Uvažujme prípad homogénnej úlohy s nulovou okrajovou podmienkou pre rýchlosť,
$v_D=0$, zadanou na celej hranici $\partial \Omega$. Dostaneme rovnice

\eq{si-discrete-problem-homogenous}{
\begin{split}
(\rho_h^{n-1}\; &\dadt v_h^n,u_h) + d(\rho_h^{n-1},v_h^{n-1}-w_h^{n-1},v_h^n,u_h)
+ a(v_h^n,u_h) \\ &=b(u_h,\pi_h^{n-1}) + (b_h^{n-1},u_h) \qquad \forall u_h \in V_h, \\ 
(\dadt &\rho_h^n,q_h) + e(\rho_h^{n-1},v_h^n,\sdtf) \\
&+\alpha(v_h^{n-1}-w_h^{n-1},\rho_h^n,\sdtf) = 0 \qquad\forall q_h \in Q_h.
\end{split}}

Pre túto úlohu dokážeme existenciu približného riešenia na ďalšej časovej
vrstve za predpokladu existencie riešenia z predchádzajúcej časovej vrstvy a
obmedzení pre veľkosť časového kroku $\tau$ a pre konštantu $\delta$.

\tu{Veta}\footnote{Viď. Feistauer et al. \cite[p. 371]{feistauer}.}
Nech $v_h^{n-1}$,$\rho_h^{n-1}$ je približné riešenie z časovej vrstvy $t_{n-1}$
také, že $\rho_h^{n-1} \ge \rho_0$, kde $\rho_0 > 0$ je kladná konštanta.
Označme 
\eq{si-proof-predpoklady1}{
K_{n-1} = \max \{ \n v_h^{n-1}\n_\infty,\n
v_h^{n-1}-w_h^{n-1}\n_\infty,\n\rho_h^{n-1}\n_\infty \}. }
Nech ďalej platí
\eq{si-proof-predpoklady2}{
\tau \le \frac{\mu\rho_0}{2K_{n-1}^4}, \quad 
\frac{3}{2}\tau \le \delta \le \frac{\mu}{4\,N\,K_{n-1}^2} .
}
Potom existuje jednoznačné riešenie $v_h^n$,\,$\rho_h^n$ úlohy
\eref{si-discrete-problem-homogenous} na časovej vrstve $t_n$.
\begin{proof}

Definujme formy

\eq{si-proof-define-forms}{
\begin{split}
\tilde a(v,\rho,u,q)&=\frac{1}{\tau}\,(\rho_h^{n-1}\,v,u) +
d(\rho_h^{n-1},v_h^{n-1}-w_h^{n-1},v,u) + a(v,u) \\
&+ \frac{1}{\tau}\,(\rho,q) + e(\rho_h^{n-1},v,q+\delta q_\beta) 
+ \alpha(v_h^{n-1}-w_h^{n-1},\rho,q+\delta q_\beta) \,, \\
F(u,q) &= b(u,\pi_h^{n-1}) + (b_h^{n-1},u) +
\frac{1}{\tau}\,(\rho_h^{n-1}v_h^{n-1},u) + \frac{1}{\tau}\, (\rho_h^{n-1},q)
\end{split}}

Úloha \eref{si-discrete-problem-homogenous} potom pre neznáme
$v=v_h^n\in V_h$, $\rho=\rho_h^n\in Q_h$ prejde v tvar

\eq{si-proof-simple-equation}{
\tilde a(v,\rho,u,q)=F(u,q) \quad \forall u\in V_h, \forall q\in Q_h.
}

Pre dôkaz existencie a jednoznačnosti riešenia tejto úlohy ukážeme pozitívnu
definitnosť formy $\tilde a$.

Použijeme Cauchyho nerovnosť a Youngovu nerovnosť v tvare
$\alpha\beta\leq\varepsilon\alpha^2+\beta^2/(4\varepsilon)$. Pre ľubovoľné
$\varepsilon_1,\ldots\varepsilon_4>0$ máme

\eq{si-proof-odhady}{
\begin{split}
\frac{1}{\tau}\,(\rho_h^{n-1}v,v)&\geq\frac{\rho_0}{\tau}\,\n v\n^2 \,, \\
|d(\rho_h^{n-1},v_h^{n-1}-w_h^{n-1},v,v)|&\leq\n\rho_h^{n-1}(v_h^{n-1}-w_h^{n-1})\n_\infty\,\n\grad
v\n\,\n v\n \\
&\leq\varepsilon_1\n\grad v\n^2 + \frac{K_{n-1}^4}{\varepsilon_1}\,\n v\n^2 \,,\\
|\alpha(v_h^{n-1}-w_h^{n-1},\rho,\rho+\delta\rho_\beta)|&=|(\rho_\beta,\rho)+\delta\n\rho_\beta\n^2|\\
&\leq\varepsilon_2\n\rho_\beta\n^2+\frac{1}{4\varepsilon_2}\,\n\rho\n^2+\delta\n\rho_\beta\n^2\\
|e(\rho_h^{n-1},v,\rho+\delta\rho_\beta)|
&\leq\varepsilon_3\n\grad v\n^2+\frac{N\,K_{n-1}^2}{4\varepsilon_3}\,\n\rho\n^2\\
&+\varepsilon_4\n\grad v\n^2+\frac{N\,K_{n-1}^2\delta^2}{4\varepsilon_4}\n\rho_\beta\n^2. 
\end{split}
}

Spojením predchádzajúcich odhadov dostaneme

\[
\begin{split}
\tilde a(v,\rho,v,\rho) \geq
&\, \left( \frac{\rho_0}{\tau}-\frac{K_{n-1}^4}{4\varepsilon_1} \right) \n v\n^2\\ 
&+(\mu-\varepsilon_1-\varepsilon_3-\varepsilon_4)\n\grad v\n^2 
+(\lambda+\mu)\n\dv v\n^2 \\
&+\left(\delta-\varepsilon_2-\frac{N\,\delta^2\,K_{n-1}^2}{4\varepsilon_4}\right)\n\rho_\beta\n^2\\
&+\left(\frac{1}{\tau}-\frac{1}{4\varepsilon_2}-\frac{N\,K_{n-1}^2}{4\varepsilon_3}\right)\n\rho\n^2.
\end{split}
\] 

Zvoľme teraz $\varepsilon_i=\mu/4$ pre $i=1,3,4$, $\varepsilon_2=\delta/2$ a
využitím predpokladu \eref{si-proof-predpoklady2} dostaneme

\[
\begin{split}
\tilde a(v,\rho,v,\rho) \geq
&\frac{\rho_0}{2\tau}\n v\n^2 +\frac{\mu}{4}\n\grad v\n^2+(\lambda+\mu)\n\dv v\n^2 \\ 
&+\frac{\delta}{4}\n\rho_\beta\n^2 + \frac{1}{2\tau}\n\rho\n^2.
\end{split}
\] 

Vidíme, že forma $\tilde a$ je pozitívne definitná a úloha
\eref{si-proof-simple-equation} má práve jedno riešenie. 

\end{proof}


Uvažujme teraz obecne nenulovú Dirichletovu okrajovú podmienku pre rýchlosť
zadanú na celej hranici $\partial \Omega$. 
Rovnice \eref{si-discrete-problem} prejdú v tvar:

\eq{si-discrete-problem-dirichlet}{
\begin{split}
(\rho_h^{n-1}\; &\dadt v_h^n,u_h) + d(\rho_h^{n-1},v_h^{n-1}-w_h^{n-1},v_h^n,u_h)
+ a(v_h^n,u_h) \\ &=b(u_h,\pi_h^{n-1}) + (b_h^{n-1},u_h) \qquad \forall u_h \in V_h, \\ 
(\dadt &\rho_h^n,q_h) + e(\rho_h^{n-1},v_h^n,\sdtf) \\
&+\alpha(v_h^{n-1}-w_h^{n-1},\rho_h^n,\sdtf) 
- \gamma \int_{\Gamma_I}{\rho_h^n v_D^n \cdot nq_h}\dA \\
&= -\gamma \int_{\Gamma_I}{\rho_D^n v_D^n \cdot nq_h}\dA \qquad\forall q_h \in Q_h, \\
&\pi_h^n = \widehat\pi(\rho_h^n),
\end{split}}

Nech $v^* \in H^1(\Omega_t)^2, v^*|_{\partial\Omega}=v_D$ je realizácia okrajovej
podmienky pre rýchlosť, hľadáme potom riešenie $v=v_h^n,\rho=\rho_h^n$ tak, že
$v-v^* \in V_h,\rho\in Q_h$. Ak označíme $v=v^*+z$ pre $z\in V_h$, úlohu
\eref{si-discrete-problem-dirichlet} môžme prepísať do tvaru 

\eq{si-simple-equation-dirichlet}{
\begin{split}
\tilde a(z+v^*,&\rho,u,q) - \gamma \int_{\Gamma_I}{\rho v_D^n \cdot nq}\dA \\
&=F(u,q) - \gamma \int_{\Gamma_I}{\rho_D^n v_D^n \cdot nq}\dA
 \quad \forall u\in V_h, \forall q\in Q_h,
\end{split}
}

kde sme využili označenie zavedené v dôkaze predchádzajúcej vety.

Definujme formy 
\[
\begin{split}
\hat a(z,\rho,u,q) &= \tilde a(z,\rho,u,q) 
- \gamma \int_{\Gamma_I}{\rho v_D^n \cdot nq}\dA, \\
\hat F(u,q) &= F(u,q) - \tilde a(v^*,\rho,u,q)
- \gamma \int_{\Gamma_I}{\rho_D^n v_D^n \cdot nq}\dA. 
\end{split}
\]

Úloha \eref{si-simple-equation-dirichlet} prejde v tvar
\eq{si-simple-equation-dirichlet2}{
\hat a(z,\rho,u,q)=\hat F(u,q) \quad \forall u\in V_h, \forall q\in Q_h
}
pre neznáme $z\in V_h$ a $\rho\in Q_h$.


\tu{Tvrdenie}\footnote{Viď Feistauer et al. \cite[p. 374]{feistauer}.}
Nech platia predpoklady predchádzajúcej vety. Predpokladajme naviac $\gamma>0$ a
$v_D\cdot n<0$  na $\Gamma_I$. Nech $v^* \in H^1(\Omega_t)^2$
je realizácia okrajovej podmienky pre rýchlosť. Potom existuje jednoznažné
riešenie $v=v_h^n$, $\rho=\rho_h^n$ úlohy
\eref{si-discrete-problem-dirichlet}, 
kde $v-v^*\in V_h, \rho\in Q_h$. 
\begin{proof}

Pre platnosť tvrdenia potrebujeme ukázať pozitívnu definitnosť formy $\hat a$.
Je $$\hat a(z,\rho,z,\rho) = \tilde a(z,\rho,z,\rho) - 
\gamma \int_{\Gamma_I}{\rho^2 v_D^n \cdot n}\dA.$$
Druhý člen na pravej strane je podľa predpokladov tvrdenia kladný, forma $\tilde a$ je
pozitívne definitná podľa predchádzajúcej vety.
\end{proof}



\chapter{Konštrukcia ALE zobrazenia pre izolovaný profil}

Pri konštrukcii ALE zobrazenia máme pomerne veľkú voľnosť. Pre praktické
počítanie sa zvykne používať metóda, v ktorej sa oblasť rozdelí na tri časti.
Prvá časť je okolie profilu, ktoré transformujeme spolu s profilom ako tuhé
teleso. Druhá časť, vypĺňajúca okolie vonkajšej hranice, sa transformuje 
identickým zobrazením. Transformácia tretej časti, ktorá sa nachádza medzi
predchádzajúcimi dvoma, bude kombináciou prvých dvoch transformácií tak, aby
sa na každej strane hladko napojila. 

\begin{figure}[h]
  \begin{center}
    \img{ale.png}
    \caption{Rôzne časti oblasti v konštrukcii ALE zobrazenia}
    \label{si-im-ale=domains}
  \end{center}
\end{figure}

Uveďme analytickú konštrukciu výsledného ALE zobrazenia. Transformácia bodov profilu
bude pre náklon $\alpha$ a vertikálnu výchylku $h$ popísaná nasledujúcim
zobrazením 

\[
H(X,Y)=
\left( \begin{array}{cc}
\cos \alpha & \sin \alpha \\
-\sin \alpha & \cos \alpha \end{array} \right) 
\cdot \left( \begin{array}{c}
X-X_{EA} \\
Y-Y_{EA} \end{array} \right)
+ \left( \begin{array}{c}
X_{EA} \\
Y_{EA} \end{array} \right)
+ \left( \begin{array}{c}
0 \\
h \end{array} \right),
\]
kde $(X,Y)$ sú referenčné súradnice transformovaného bodu a 
$(X_{EA},Y_{EA})$ sú referenčné súradnice osi okolo ktorej sa profil otáča.\\
Identické zobrazenie budeme značiť $Id(X,Y)=(X,Y)$. 
Definujme ďalej váhové funkcie $\xi$, $\theta$ v závislosti na
$R=R(X,Y),R_1,R_2$ predpisom

\[
\begin{split}
\xi(R)&=\min\left(\max\left(0,\frac{R-R_1}{R_2-R_1}\right),1\right) \\
\theta(R)&=\frac{1}{2}\left(\cos(\xi(R)\,\pi)+1\right)
\end{split}
\]
 
 Pomocou predchádzajúcich funkcií môžme zostrojiť výsledné ALE zobrazenie
 predpisom 
 
 $$\mathcal A_t(X,Y)=\theta(R)\,H(X,Y)+(1-\theta(R))\,Id(X,Y)$$
 
 Pre zobrazenie $R=R(X,Y)$ máme opäť možnosť voľby (určí sa ním tvar oblasti,
 ktorá sa bude transformovať ako pevné teleso). Pre letecký profil je výhodné
 použiť elipsu, pre ktorú tak máme 
 $R(X,Y)=\sqrt{\frac{X^2}{a^2}+\frac{Y^2}{b^2}}$, kde $a,b>0$ sú jej poloosi.
 Dodajme ešte, že konštanty $R_1$ a $R_2$ určia veľkosť oblasti $\Omega_3$.  
 
\chapter{Riešenie diskrétneho problému}

Našou úlohou je zostrojiť priestory $Q_h$ a $X_h$, v ktorých budeme hľadať
približné riešenie. K tomu použijeme Galerkinovu metódu, tj. prevedieme
trianguláciu oblasti na konečné elementy $\mathcal T_h$ a definujeme na tejto
triangulácii bázové funkcie, ktoré vytvoria bázy príslušných priestorov. 

Keď máme k dispozícii približné priestory $Q_h$ a $X_h$, môžme pristúpiť k
riešeniu približného problému. Pre rýchlosť a pre tlak vytvoríme v každom
časovom intervale sústavu lineárnych rovníc, ktorých riešením dostaneme
približné hodnoty $v_h^n$ a $\rho_h^n$. Voľba numerickej schémy
\eref{si-discrete-problem} nám umožňuje oddeliť rovnice pre rýchlosť a
pre tlak na dve samostatné sústavy a každú z nich riešiť oddelene.


\section{Triangulácia oblasti}

Na trianguláciu oblasti sme použili polygonálnu trianguláciu, letecký profil sme
aproximovali po častiach lineárnou lomenou krivkou. Výpočtová oblasť je tak
tvorená trojuholníkovými konečnými elementami $K\in\mathcal T_h$ (viď \iref{si-im-mesh}).

Na konštrukciu a adaptívne zjemňovanie triangulácie bol využitý software
ANGENER\footnote{http://www.karlin.mff.cuni.cz/~dolejsi/angen/angen.htm}. 

\begin{figure}[h]
  \begin{center}
    \img{mesh.png}
    \caption{Príklad izotropnej siete}
    \label{si-im-mesh}
  \end{center}
\end{figure}


\section{Voľba bázových funkcií}


Pri riešení úlohy sme použili lineárne konečné prvky pre tlak aj pre rýchlosť.

Popíšeme najprv voľbu bázových funkcií priestoru $Q_h$. Pre každý vrchol~$P$
triangulácie $\mathcal T_h$ definujeme jednu bázovú funkciu $q_h^P$ hodnotou vo
vrchole $P$: $q_h^P(P) = 1$, pre každý ďalší vrchol $P'$ definujeme
$q_h^P(P') = 0$. Keďže $q_h^P$ je lineárna na každom elemente, je týmto určená jednoznačne. 
Nosičom takto definovanej bázovej funkcie sú len elementy obsahujúce bod $P$ ako
svoj vrchol. Zároveň je daná aj dimenzia priestoru $Q_h$. Ak označíme počet vrcholov
triangulácie $N_h = \# \mbox{vrcholov} \mathcal T_h$, potom $\dim Q_h = N_h$.

Bázové funkcie priestoru $X_h$ volíme podobne. Pre každý vrchol $P$ definujeme
dve bázové funkcie $u_{hx}^P$ a $u_{hy}^P$, kde $u_{hx}^P = (q_h^P,0)$ a 
$u_{hy}^P = (0,q_h^P)$, kde $q_h^P$ je príslušná bázová funkcia priestoru $Q_h$.
Nosičom takto definovaných bázových funkcií sú opäť len elementy obsahujúce
vrchol $P$ a dimenzia priestoru $\dim X_h = 2N_h$.

Vrcholy triangulácie a im odpovedajúce bázové funkcie očíslujme:
$$P_i,q_h^i,u_{hx}^i,u_{hy}^i,\;i=1,\ldots N_h.$$

Približné riešenia $\rho_h$ a $v_h$ sú lineárnymi kombináciami príslušných bázových
funkcií: 
\eq{si-rozpis-pribl-ries-do-bazy}{
\begin{split}
\rho_h &= \sum_{i=1}^{N_h} \alpha_i \, q_h^i \,, \\
v_h = &\sum_{i=1}^{N_h} (\beta_i^x \, u_{hx}^i + \beta_i^y \, u_{hy}^i) \,. 
\end{split}}

Hodnota približného riešenia $\rho_h$ vo vrchole $P_i$ je tak daná koeficientom
$\alpha_i$, podobne hodnota x-ovej a y-ovej zložky približného riešenia $v_h$ v
uvažovanom vrchole je daná koeficientami $\beta_i^x$ a $\beta_i^y$.

Pre ďalšie zjednodušenie zápisu použime označenie
\[
\begin{array}{lll}
\beta_i=\beta_k^x, & u_h^i=u_{hx}^k, & \mbox{pre } i=2k-1 \\
\beta_i=\beta_k^y, & u_h^i=u_{hy}^k, & \mbox{pre } i=2k, \quad k=1\ldots N_h.
\end{array}
\]

Približné riešenie $v_h$ tak môžme zapísať ako lineárnu kombináciu
$$v_h = \sum_{i=1}^{2N_h} \beta_i \, u_h^i.$$

\section{Zostavenie sústav lineárnych rovníc pre výpočet rýchlosti a tlaku}

Popíšeme zostavenie sústav lineárnych rovníc pre výpočet rýchlosti a tlaku pre
prípad pevnej oblasti. V tomto prípade sa rovnice \eref{si-discrete-problem}
zjednodušia, $\DADt f(y,t) = \ddt f(y,t),$ a $w = 0$.

Uvažované rovnice tak majú nasledujúci tvar

\eq{si-discrete-problem-without-ALE}{
\begin{split}
(\rho_h^{n-1}\; &d_t v_h^n,u_h) + d(\rho_h^{n-1},v_h^{n-1},v_h^n,u_h) + a(v_h^n,u_h) \\
&=b(u_h,\pi_h^{n-1}) + (b_h^{n-1},u_h) + \int_{\Gamma_O}{\pi_{ref}\,u_h \cdot
n}\dA \qquad \forall u_h \in V_h, \\ 
(d_t &\rho_h^n,q_h) + e(\rho_h^{n-1},v_h^n,\sdtf) \\
&+\alpha(v_h^{n-1},\rho_h^n,\sdtf) 
- \gamma \int_{\Gamma_I}{\rho_h^n v_D^n \cdot nq_h}\dA \\
&= -\gamma \int_{\Gamma_I}{\rho_D^n v_D^n \cdot nq_h}\dA \qquad\forall q_h \in Q_h, \\
&\pi_h^n = \widehat\pi(\rho_h^n),
\end{split}}

kde $d_t \rho_h^n = (\rho_h^n - \rho_h^{n-1})/\tau$, $d_t v_h^n = (v_h^n -
v_h^{n-1})/\tau$. 

Rozpísaním diferencií $d_t\rho_h^n$, $d_t v_h^n$ a rozpísaním $\rho_h^n$,
$v_h^n$ ako lineárnych kombinácií bázových funkcií, kde píšeme $u_i=u_h^i$ a 
$q_i=q_h^i$, dostaneme 

\eq{si-discrete-problem-without-ALE-rozpisane}{
\begin{split}
\sum_{i=1}^{2N_h} &\,\beta_i \, \left( \fr{1}{\tau}\,(\rho_h^{n-1}\,u_i,u_j) + 
d(\rho_h^{n-1},v_h^{n-1},u_i,u_j) + a(u_i,u_j) \right) \\
&= \fr{1}{\tau}\,(\rho_h^{n-1}\,v_h^{n-1},u_j) +
b(u_j,\pi_h^{n-1}) + (b_h^{n-1},u_j) \\
&+\int_{\Gamma_O}{\pi_{ref}\,u_j \cdot n}\dA \qquad \forall j=1\ldots 2N_h,
\\ \sum_{i=1}^{N_h} &\,\alpha_i \, \left( \fr{1}{\tau}\,(q_i,q_j) +
\alpha(v_h^{n-1},q_i,\sdf{j}) 
- \gamma \int_{\Gamma_I}{q_i v_D^n \cdot nq_j}\dA \,\right) \\
&= \fr{1}{\tau}\,(\rho_h^{n-1},q_j) - e(\rho_h^{n-1},v_h^n,\sdf{j}) \\
&-\gamma \int_{\Gamma_I}{\rho_D^n v_D^n \cdot nq_j}\dA \qquad\forall j=1\ldots N_h,
\\ &\pi_h^n = \widehat\pi(\rho_h^n),
\end{split}}


Uvedené rovnice tvoria systém $2N_h$ lineárnych rovníc pre $2N_h$ neznámych
koeficientov $\beta_i$ a systém $N_h$ lineárnych rovníc pre $N_h$ neznámych
koeficientov $\alpha_i$. 

Ich riešením dostaneme priamo hodnoty približného riešenia $v_h$ a $\rho_h$ v
príslušných vrcholoch.
  
Výhodou použitej Galerkinovej metódy je, že každá bázová funkcia má malý nosič.
Pri výpočte jednotlivých koeficientov matíc tuhosti a vektorov pravých strán 
nám tak stačí počítať len diagonálne koeficienty a koeficienty odpovedajúce
vrcholom triangulácie, ktoré sú spojené stranou jedného z elementov. Ostatné
koeficienty matíc tuhosti budú nulové, nakoľko pre takéto dvojice je na každom
prvku vždy aspoň jedna z testovacích funkcií $u_i, u_j$ resp. $q_i,q_j$ nulová.   
Dostávame tak pre rýchlosť aj pre tlak sústavy s riedkymi maticami tuhosti. 

\section{Riešenie lineárnych sústav}

Na riešenie lineárnych sýstav \eref{si-discrete-problem-without-ALE-rozpisane}
sme použili priamy riešič pre riedke matice
PARDISO\footnote{http://www.pardiso-project.org/index.html}. Tento riešič pracuje pre
symetrické aj nesymetrické sústavy a je vhodný pre paralelné
spúštanie. Jeho veľkou výhodou je pomerne malá pamäťová náročnosť a vysoká
rýchlosť (výpočty tak mohli byť prevádzané na bežne dostupnom PC v rozumnom čase).

\chapter{Príklady}

Previedli sme niektoré numerické experimenty pre prípad kanála, vnútri ktorého
je nepohybujúci sa profil. Zmena uhlu nábehu profilu pre rôzne experimenty 
je riešená zadávaním rôznej okrajovej podmienky pre rýchlosť.  

Voľba oblasti je znázornená na obrázku \ref{si-im-oblast}
\begin{figure}[h]
  \begin{center}
    \img{oblast.png}
    \caption{Voľba výpočtovej oblasti}
    \label{si-im-oblast}
  \end{center}
\end{figure}

Prvý experiment sme previedli s nasledovnými vstupnými parametrami:\\\\
$\mu = 1$ \\
$\lambda = -\frac{2}{3}\mu$ \\
$\rho_{ref} = 1$ \\
$\pi(\rho) = 1$ \\
$v_{ref} = (1,0)$ \\
$\tau = 0.001$ \\
$\delta = \frac{3}{2}\tau$ \\

Pre dané parametre dostávame veľkosť Reynoldsovho čísla
$$Re=\frac{\rho_{ref}\,v_{ref}}{\mu} = 1$$ (uvažovaná charakteristická dĺžka
profilu je $l=1$).  


Počiatočné podmienky pre rýchlosť a tlak volíme $v_0=v_{ref}$, resp.
$\rho_0=\rho_{ref}$. Okrajovú podmienku pre rýchlosť na vstupe a na hornej a
spodnej časti kanála volíme $v_D=v_{ref}$, na profile predpokladáme nulovú
rýchlosť(vzhľadom k profilu). Hustotu predpisujeme na vstupe hodnotou $\rho_D=\rho_{ref}$. 


Z výsledkov experimentu vidno spojitú zmenu rozdelenia hustoty (a teda aj tlaku) v
čase aj v priestore. V prednej časti profilu podľa očakávania dôjde k zvýšeniu
tlaku, v chvostovej časti dochádza k jeho zníženiu. Nábeh trvá približne pol
sekundy, potom sa riešenie stabilizuje a mení sa len minimálne.

\begin{figure}[h]
  \begin{center}
    \img{mesh-adapt.png}
    \caption{Adaptovaná sieť zachycujúca zmenu rýchlosti v okolí profilu}
    \label{si-im-mesh-adapt}
  \end{center}
\end{figure}

\begin{figure}[h]
  \begin{center}
    \img{hustota1.png}
    \caption{Rozdelenie hustoty pre prípad $\alpha=0$ v čase $t=1$}
    \label{si-im-hustota1}
  \end{center}
\end{figure}

\begin{figure}[h]
  \begin{center}
    \img{hustota2.png}
    \caption{Rozdelenie hustoty pre prípad $\alpha=\pi/10$ v čase $t=0.4$}
    \label{si-im-hustota2}
  \end{center}
\end{figure}

\begin{figure}[h]
  \begin{center}
    \img{prudnice1.png}
    \caption{Zobrazenie prúdnic rýchlosti pre prípad $\alpha=0$ v čase $t=1$}
    \label{si-im-prudnice1}
  \end{center}
\end{figure}

\begin{figure}[h]
  \begin{center}
    \img{prudnice2.png}
    \caption{Zobrazenie prúdnic rýchlosti pre prípad $\alpha=\pi/10$ v čase $t=0.4$}
    \label{si-im-prudnice2}
  \end{center}
\end{figure}

Prevádzali sme aj experimenty pre reálne dáta odpovedajúce prúdeniu vzduchu,
tlak sme počítali zo vzťahu $\pi(\rho)=c\,\rho^\kappa$. Vstupné parametre boli nasledovné:\\\\
$\mu = 1.8\cdot 10^{-5}$ \\
$\lambda = -\frac{2}{3}\mu$ \\
$\rho_{ref} = 1.20$ \\
$\pi(\rho)=c\,\rho^\kappa$ \\
$c=78500$ \\
$\kappa=1.4$ \\
$\tau = 0.001$ \\
$\delta = \frac{3}{2}\tau$ \\

Referenčnú rýchlosť sme volili v rôznych experimentoch voľbou Reynoldsovho čísla
z rozmedzia $Re=10^1\ldots10^6$. Pri takýchto vstupných dátach prestal výpočet
fungovať. Pomerne rýchlo (niekoľko časových krokov) dochádzalo v riešení k
veľkému rozdielu medzi minimálnym a maximálnym tlakom, ako aj medzi minimálnou a
maximálnou rýchlosťou a spočítané riešenie nemalo žiaden fyzikálny význam.
Skúšali sme meniť parameter $\delta$ v rozmedzí danom predpokladmi
\eref{si-proof-predpoklady2}, ale bezvýsledne. 

Z predpokladov \eref{si-proof-predpoklady2} vyplýva aj ohraničenie pre časový
krok.\\ 
Pre $Re=10^3$ dostávame horný odhad pre časový krok rovný $\tau=10^{-5}$,
pre vyššie Reynoldove čísla možná veľkosť časového kroku rapídne klesá (pre
$Re=10^6$ je už $\tau=10^{-9}$). 
Previedli sme výpočty pre časové kroky $\tau=0.0001$ a $\tau=0.00001$, ale k
získaniu rozumných výsledkov to neviedlo.   
Pre takéto malé časové kroky zrejme vo výpočte začali prevládať zaokrúhľovacie
chyby a tým bolo znehodnotené. Aj v prípade eliminácie zaokrúhľovacích chýb
použitím presnejšej aritmetiky by sme však narazili na nemožnosť spočítať
riešenie na dlhšom časovom intervale v rozumnej dobe, nakoľko časový krok pre
jednu iteráciu je veľmi malý.

\chapter{Diskusia}

Pri numerickom experimentovaní na reálnych vstupných dátach úlohy sme rýchlo
narazili na obmedzenia pre časový krok vyplývajúce z dôkazu existencie
približného riešenia. Problém sa dá čiastočne vyriešiť použitím presnejšej
aritmetiky výpočtu, ale len za cenu neúmerného predĺženia výpočtu. 

Lepšie riešenie spočíva v zvýšení rádu použitých konečných prvkov, nakoľko v
experimentoch používame len lineárne prvky $P^1/P^1$ pre rýchlosť aj pre tlak.
Tento krok by mohol pomôcť hlavne k presnejšiemu výpočtu rýchlosti, s ktorou
boli najväčšie problémy. 

Ďalšie zlepšenie je možné v tvorbe rôznych triangulácií pre rýchlosť a pre tlak.
V prevedených experimentoch sme počítali tlak pre sieť adaptovanú na rozloženie
rýchlosti. Takáto sieť je veľmi hustá v okolí profilu, pretože tam dochádza k
veľkým zmenám rýchlosti. Lenže tlak sa v okolí profilu správa spojite a
počítanie na hustej sieti tak zbytočne predĺži dobu výpočtu. 

Paradoxne najmenšie problémy boli s riešením vzniknutých lineárnych sústav.
Použitý software pracoval efektívne a môžme ho doporučiť. Experimentovali sme
aj s veľmi hustými sieťami s počtom vrcholov $N_h=40000$, pre ktoré vzniknutá
matica tuhosti pre rýchlosť mala $80000$ riadkov, a ani pre takéto dáta nebolo
riešenie lineárnej sústavy úzkym hrdlom výpočtu.
 
\chapter{Záver}

V tejto práci sme ukázali odvodenie Navier-Stokesových rovníc modelu
stlačiteľnej barotropickej tekutiny a popisu metódy pre úlohu s premennou
hranicou. 
Ukázali sme existenciu a jednoznačnosť približného riešenia pre prípad
Dirichletovej okrajovej podmienky.
Ďalej sme previedli numerické experimenty pre prípad pevnej hranice a overili
sme nutnosť splnenia predpokladov pre voľbu časového kroku vyplývajúcich z
dôkazu existencie riešenia.
