\chapter{Existence of approximate solution}
 
Assume homogenous Dirichlet condition $v_D=0$ given on the whole boundary $\partial \Omega$. We obtain equations

\eq{si-discrete-problem-homogenous}{
\begin{split}
(\rho_h^{n-1}\; &\dadt v_h^n,u_h) + d(\rho_h^{n-1},v_h^{n-1}-w_h^{n-1},v_h^n,u_h)
+ a(v_h^n,u_h) \\ &=b(u_h,\pi_h^{n-1}) + (b_h^{n-1},u_h) \qquad \forall u_h \in V_h, \\ 
(\dadt &\rho_h^n,q_h) + e(\rho_h^{n-1},v_h^n,\sdtf) \\
&+\alpha(v_h^{n-1}-w_h^{n-1},\rho_h^n,\sdtf) = 0 \qquad\forall q_h \in Q_h.
\end{split}}

For this problem, we will prove existence of approximate solution on next time level under assumptions of existence 
of approximate solution from previous time level and constraints for time step
$\tau$ and for constant $\delta$.

\tu{Theorem}\footnote{See. Feistauer et al. \cite[p. 371]{feistauer}.}
Let $v_h^{n-1}$,$\rho_h^{n-1}$ be approximate solution in time level $t_{n-1}$,
such that $\rho_h^{n-1} \ge \rho_0$, where $\rho_0 > 0$ is a positive constant.
Denote 
\eq{si-proof-predpoklady1}{
K_{n-1} = \max \{ \n v_h^{n-1}\n_\infty,\n
v_h^{n-1}-w_h^{n-1}\n_\infty,\n\rho_h^{n-1}\n_\infty \}. }
Further, let
\eq{si-proof-predpoklady2}{
\tau \le \frac{\mu\rho_0}{2K_{n-1}^4}, \quad 
\frac{3}{2}\tau \le \delta \le \frac{\mu}{4\,N\,K_{n-1}^2} .
}
Then there exists a unique solution $v_h^n$,\,$\rho_h^n$ of problem
\eref{si-discrete-problem-homogenous} on time level $t_n$.
\begin{proof}

Denote forms

\eq{si-proof-define-forms}{
\begin{split}
\tilde a(v,\rho,u,q)&=\frac{1}{\tau}\,(\rho_h^{n-1}\,v,u) +
d(\rho_h^{n-1},v_h^{n-1}-w_h^{n-1},v,u) + a(v,u) \\
&+ \frac{1}{\tau}\,(\rho,q) + e(\rho_h^{n-1},v,q+\delta q_\beta) 
+ \alpha(v_h^{n-1}-w_h^{n-1},\rho,q+\delta q_\beta) \,, \\
F(u,q) &= b(u,\pi_h^{n-1}) + (b_h^{n-1},u) +
\frac{1}{\tau}\,(\rho_h^{n-1}v_h^{n-1},u) + \frac{1}{\tau}\, (\rho_h^{n-1},q)
\end{split}}

Problem \eref{si-discrete-problem-homogenous} with unknowns
$v=v_h^n\in V_h$, $\rho=\rho_h^n\in Q_h$ can be written in a form

\eq{si-proof-simple-equation}{
\tilde a(v,\rho,u,q)=F(u,q) \quad \forall u\in V_h, \forall q\in Q_h.
}

To prove existence and uniqueness, we show that $\tilde a$ is positively definite. 

We will use Cauchy's inequality and Young's inequality in a form
$\alpha\beta\leq\varepsilon\alpha^2+\beta^2/(4\varepsilon)$. For arbitrary
$\varepsilon_1,\ldots\varepsilon_4>0$ we can write

\eq{si-proof-odhady}{
\begin{split}
\frac{1}{\tau}\,(\rho_h^{n-1}v,v)&\geq\frac{\rho_0}{\tau}\,\n v\n^2 \,, \\
|d(\rho_h^{n-1},v_h^{n-1}-w_h^{n-1},v,v)|&\leq\n\rho_h^{n-1}(v_h^{n-1}-w_h^{n-1})\n_\infty\,\n\grad
v\n\,\n v\n \\
&\leq\varepsilon_1\n\grad v\n^2 + \frac{K_{n-1}^4}{\varepsilon_1}\,\n v\n^2 \,,\\
|\alpha(v_h^{n-1}-w_h^{n-1},\rho,\rho+\delta\rho_\beta)|&=|(\rho_\beta,\rho)+\delta\n\rho_\beta\n^2|\\
&\leq\varepsilon_2\n\rho_\beta\n^2+\frac{1}{4\varepsilon_2}\,\n\rho\n^2+\delta\n\rho_\beta\n^2\\
|e(\rho_h^{n-1},v,\rho+\delta\rho_\beta)|
&\leq\varepsilon_3\n\grad v\n^2+\frac{N\,K_{n-1}^2}{4\varepsilon_3}\,\n\rho\n^2\\
&+\varepsilon_4\n\grad v\n^2+\frac{N\,K_{n-1}^2\delta^2}{4\varepsilon_4}\n\rho_\beta\n^2. 
\end{split}
}

Putting previous estimates together, we get

\[
\begin{split}
\tilde a(v,\rho,v,\rho) \geq
&\, \left( \frac{\rho_0}{\tau}-\frac{K_{n-1}^4}{4\varepsilon_1} \right) \n v\n^2\\ 
&+(\mu-\varepsilon_1-\varepsilon_3-\varepsilon_4)\n\grad v\n^2 
+(\lambda+\mu)\n\dv v\n^2 \\
&+\left(\delta-\varepsilon_2-\frac{N\,\delta^2\,K_{n-1}^2}{4\varepsilon_4}\right)\n\rho_\beta\n^2\\
&+\left(\frac{1}{\tau}-\frac{1}{4\varepsilon_2}-\frac{N\,K_{n-1}^2}{4\varepsilon_3}\right)\n\rho\n^2.
\end{split}
\] 

Let $\varepsilon_i=\mu/4$ for $i=1,3,4$, $\varepsilon_2=\delta/2$ and
using \eref{si-proof-predpoklady2} we obtain

\[
\begin{split}
\tilde a(v,\rho,v,\rho) \geq
&\frac{\rho_0}{2\tau}\n v\n^2 +\frac{\mu}{4}\n\grad v\n^2+(\lambda+\mu)\n\dv v\n^2 \\ 
&+\frac{\delta}{4}\n\rho_\beta\n^2 + \frac{1}{2\tau}\n\rho\n^2.
\end{split}
\] 

Now, it is obvious that form $\tilde a$ is positively definite. Thus problem
\eref{si-proof-simple-equation} has exactly one solution. 

\end{proof}


Let us assume generally nonzero Dirichlet boundary condition for velocity given on the whole boundary $\partial \Omega$. 
Problem \eref{si-discrete-problem} will have a form

\eq{si-discrete-problem-dirichlet}{
\begin{split}
(\rho_h^{n-1}\; &\dadt v_h^n,u_h) + d(\rho_h^{n-1},v_h^{n-1}-w_h^{n-1},v_h^n,u_h)
+ a(v_h^n,u_h) \\ &=b(u_h,\pi_h^{n-1}) + (b_h^{n-1},u_h) \qquad \forall u_h \in V_h, \\ 
(\dadt &\rho_h^n,q_h) + e(\rho_h^{n-1},v_h^n,\sdtf) \\
&+\alpha(v_h^{n-1}-w_h^{n-1},\rho_h^n,\sdtf) 
- \gamma \int_{\Gamma_I}{\rho_h^n v_D^n \cdot nq_h}\dA \\
&= -\gamma \int_{\Gamma_I}{\rho_D^n v_D^n \cdot nq_h}\dA \qquad\forall q_h \in Q_h, \\
&\pi_h^n = \widehat\pi(\rho_h^n),
\end{split}}

Let $v^* \in H^1(\Omega_t)^2, v^*|_{\partial\Omega}=v_D$ be a realization of the boundary condition, 
we find solution $v=v_h^n,\rho=\rho_h^n$, such that
$v-v^* \in V_h,\rho\in Q_h$. Let $v=v^*+z$ for $z\in V_h$, we may write
problem \eref{si-discrete-problem-dirichlet} in a form 

\eq{si-simple-equation-dirichlet}{
\begin{split}
\tilde a(z+v^*,&\rho,u,q) - \gamma \int_{\Gamma_I}{\rho v_D^n \cdot nq}\dA \\
&=F(u,q) - \gamma \int_{\Gamma_I}{\rho_D^n v_D^n \cdot nq}\dA
 \quad \forall u\in V_h, \forall q\in Q_h,
\end{split}
}

where we use denotation from the proof of previous theorem.

Define following forms
\[
\begin{split}
\hat a(z,\rho,u,q) &= \tilde a(z,\rho,u,q) 
- \gamma \int_{\Gamma_I}{\rho v_D^n \cdot nq}\dA, \\
\hat F(u,q) &= F(u,q) - \tilde a(v^*,\rho,u,q)
- \gamma \int_{\Gamma_I}{\rho_D^n v_D^n \cdot nq}\dA. 
\end{split}
\]

Problem \eref{si-simple-equation-dirichlet} will have this form
\eq{si-simple-equation-dirichlet2}{
\hat a(z,\rho,u,q)=\hat F(u,q) \quad \forall u\in V_h, \forall q\in Q_h
}
for unknowns $z\in V_h$ and $\rho\in Q_h$.


\tu{Theorem}\footnote{See Feistauer et al. \cite[p. 374]{feistauer}.}
Let assumptions from previous theorem hold. Moreover, assume $\gamma>0$ and let
$v_D\cdot n<0$ on $\Gamma_I$. Let $v^* \in H^1(\Omega_t)^2$
be the realization of boundary condition for velocity. Then, there exists a unique solution 
$v=v_h^n$, $\rho=\rho_h^n$ of the problem
\eref{si-discrete-problem-dirichlet}, 
where $v-v^*\in V_h, \rho\in Q_h$. 
\begin{proof}

We need to show that $\hat a$ is positively definite.
We may write $$\hat a(z,\rho,z,\rho) = \tilde a(z,\rho,z,\rho) - 
\gamma \int_{\Gamma_I}{\rho^2 v_D^n \cdot n}\dA.$$
Second term in righthand side is positive (under the assumptions of the theorem), form  $\tilde a$ is
positively definite by the previous theorem.
\end{proof}

